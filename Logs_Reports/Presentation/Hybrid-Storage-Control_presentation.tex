% Copyright 2004 by Till Tantau <tantau@users.sourceforge.net>.
%
% In principle, this file can be redistributed and/or modified under
% the terms of the GNU Public License, version 2.
%
% However, this file is supposed to be a template to be modified
% for your own needs. For this reason, if you use this file as a
% template and not specifically distribute it as part of a another
% package/program, I grant the extra permission to freely copy and
% modify this file as you see fit and even to delete this copyright
% notice. 

\documentclass{beamer}
\makeatletter
\renewcommand*\env@matrix[1][*\c@MaxMatrixCols c]{%
	\hskip -\arraycolsep
	\let\@ifnextchar\new@ifnextchar
	\array{#1}}
\makeatother
%%\usepackage[style=verbose,backend=biber]{biblatex}
%%\addbibresource{ref.bib}
\setbeamerfont{footnote}{size=\tiny}
\beamertemplatenavigationsymbolsempty
% There are many different themes available for Beamer. A comprehensive
% list with examples is given here:
% http://deic.uab.es/~iblanes/beamer_gallery/index_by_theme.html
% You can uncomment the themes below if you would like to use a different
% one:
%\usetheme{AnnArbor}
%\usetheme{Antibes}
%\usetheme{Bergen}
%\usetheme{Berkeley}
%\usetheme{Berlin}
\usetheme{Boadilla}
%\usetheme{boxes}
\usepackage{color}
\usepackage{tikz}
\usepackage{caption}
\usepackage{graphicx}
%\usepackage{figure}
%\usetheme{CambridgeUS}
%\usetheme{Copenhagen}
%\usetheme{Darmstadt}
%\usetheme{default}
\DeclareMathOperator*{\argmin}{argmin}
%\usetheme{Frankfurt}
%\usetheme{Goettingen}
%\usetheme{Hannover}
%\usetheme{Ilmenau}
%\usetheme{JuanLesPins}
%\usetheme{Luebeck}
%\usetheme{Madrid}
%\usetheme{Malmoe}
%\usetheme{Marburg}
%\usetheme{Montpellier}
%\usetheme{PaloAlto}
%\usetheme{Pittsburgh}
%\usetheme{Rochester}
%\usetheme{Singapore}
%\usetheme{Szeged}
%\usetheme{Warsaw}
\usepackage{amsmath}
% correct bad hyphenation here
\hyphenation{op-tical net-works semi-conduc-tor}


%\usepackage{soul}
\usepackage{color}




%Use Tiks to plot graph
\usepackage{pgf}
\usepackage{tikz}
\usetikzlibrary{arrows,automata}
%figure package
\usepackage{float}
\title{}

% A subtitle is optional and this may be deleted
%\subtitle{ECE1094 Project Presentation}

\author{Zhongbin Huang}
% - Give the names in the same order as the appear in the paper.
% - Use the \inst{?} command only if the authors have different
%   affiliation.

% \institute[Universities of Toronto] % (optional, but mostly needed)
% {
%   \inst{1}%
%   Department of Electrc\\
%   University of Toronto
%   \and
%   \inst{2}%
%   Department of Theoretical Philosophy\\
%   University of Elsewhere}
% - Use the \inst command only if there are several affiliations.
% - Keep it simple, no one is interested in your street address.

\date{Oct 26, 2017}
% - Either use conference name or its abbreviation.
% - Not really informative to the audience, more for people (including
%   yourself) who are reading the slides online

%\subject{Theoretical Computer Science}
% This is only inserted into the PDF information catalog. Can be left
% out. 

% If you have a file called "university-logo-filename.xxx", where xxx
% is a graphic format that can be processed by latex or pdflatex,
% resp., then you can add a logo as follows:

% \pgfdeclareimage[height=0.5cm]{university-logo}{university-logo-filename}
% \logo{\pgfuseimage{university-logo}}

% Delete this, if you do not want the table of contents to pop up at
% the beginning of each subsection:
% \AtBeginSubsection[]
% {
%   \begin{frame}<beamer>{Outline}
%     \tableofcontents[currentsection,currentsubsection]
%   \end{frame}
% }

% Let's get started
\begin{document}
	
	\begin{frame}
	\titlepage
\end{frame}

%\begin{frame}{Outline}
%\begin{itemize}
%	\item System and Attack Model
%	\item Undetectable Attack, Unidentifiable Attack, {\color{red}Structurally Undetectable Attack}
%	\item An Example Related to My Current Work
%	\item Monitoring Design
%\end{itemize}
%\end{frame}
 \begin{frame}{Outline}
   \tableofcontents
   % You might wish to add the option [pausesections]
 \end{frame}

% Section and subsections will appear in the presentation overview
% and table of contents.
\section{Background}

\section{LTI system model}

\begin{frame}{LTI system model}
\begin{itemize}
\item {System and Attack Model %%/footcite{pasqualetti2015control}
	\begin{equation}
	\begin{aligned}
	E\dot x(t) &= Ax(t) +{\color{red}B u(t)} \\
	y(t)&=Cx(t) + {\color{red}D u(t)}\\
	\end{aligned}
	\end{equation}}
where $A$ is the system matrix, $B$ is the state attack matrix,$C$ is the measurement output matrix, $D$ is measurement attack matrix, $x$ is the system state vector, $y$ is the measurement output vector and $u$ is the attack vector. All the quantities are in proper dimensions.
\item Define {\color{red}attack set} $K$ as $(B_K,D_K)$, the sparsity pattern.
\item Q: Is the attack $(B,D,u(t))$ detectable/identifiable from output y(t) ?
\end{itemize}
\end{frame}
\section{Undetectable/Unidentifiable Attack}
\begin{frame}{Undetectable Attack}
%{\color{Caution:}} Be very careful when talking about the conterp
\begin{tikzpicture}
\begin{scope}[shift={(0cm,0cm)}, fill opacity=1]
\draw[fill=none, draw = black] (-2,0) circle (3);
\draw[fill=none, draw = black] (2,0) circle (3);
%\node at (0,4) (A) {\large\textbf{A}};
\node at (-2.5,1) [draw, text width=3cm]{Normal Operating Condition\\$y(x_2,0,t)$};
%\node (example-textwidth-1) [draw, text width=3cm]{example \\ example};
\node at (2.5,1) [draw, text width=2cm]{Detectable Attacks\\$y(x_1,u_K,t)$};
\node at (0,-1) [draw, text width=2cm]{Undetectable Attacks};
\end{scope}

\end{tikzpicture}
\end{frame}



\section{Example}
\begin{frame}{A Numerical Example}
\begin{itemize}
\item{Define the following\\
$A=\begin{bmatrix}[c c c c c c]
0  & 0  & 0  & 1  & 0  & 0 \\
0  & 0  & 0  & 0  & 1  & 0 \\
0  & 0  & 0  & 0  & 0  & 1 \\
-2 & 1  & 0  & 0  & 0  & 0 \\
1  & -2 & 1  & 0  & 0  & 0 \\
0  & 1  & -2 & 0  & 0  & 0 \\
\end{bmatrix}$,
$B_K=\begin{bmatrix}[c c]
0  & 0  \\
1  & 0  \\
0  & 0  \\
0  & 0  \\
0  & 1  \\
0  & 0  \\
\end{bmatrix}$,
$C=\begin{bmatrix}[c c c c c c]
1  & 0  & 0  & 0  & 0  & 0 \\
0  & 0  & 0  & 4  & 0  & 0 
\end{bmatrix}$, $B=I$, $E=I$.}
\item {Goal: To design $F$ and make the attack set $K$ structurally detectable with the closed loop system $A+BF$, {\color{red}from output $y$}}.
\end{itemize}
\end{frame}
\begin{frame}{A Numerical Example}
\begin{itemize}
\item After some numerical operation, I get
$A+BF=\begin{bmatrix}[c c c c c c]
-1.0573  & 0.45130  & 0  & 1  & 0  & 0 \\
0  & 0  & 0  & 0  & 1  & 0 \\
0  & 0  & 0  & 0  & 0  & 1 \\
-2 & 1  & 0  & 0  & -2.769  & 0 \\
1  & -2 & 1  & 0  & 0  & 0 \\
0  & 1  & -2 & 0  & 0  & 0 \\
\end{bmatrix}$
\end{itemize}
\end{frame}
\begin{frame}{A Numerical Example}
\begin{figure}[!htb]\centering
\begin{minipage}{0.49\linewidth}
\scalebox{0.6}{\begin{tikzpicture}[->,>=stealth',shorten >=1pt,node distance=1.8cm, semithick]
\tikzstyle{every state}=[fill=none,draw=black,text=black,minimum size=1cm]

\node[state]         (X1)                    {$X_1$};
\node[state]         (X2) [right of=X1]      {$X_2$};
\node[state]         (X3) [right of=X2]      {$X_3$};
\node[state]         (X4) [below of=X1]       {$X_4$};
\node[state]         (X5) [below of=X2]       {$X_5$};
\node[state]         (X6) [below of=X3]       {$X_6$};
\node[state]         (Y1) [left of=X1]       {$Y_1$};
\node[state]         (Y2) [left of=X4]       {$Y_2$};
\node[state]         (W1) [above right of=X2] {$W_1$};
\node[state]         (W2) [below right of=X5] {$W_2$};
\path (X1) edge              node {} (X4)
edge              node {} (X5)
edge              node {} (Y1)
(X2) edge              node {} (X4)
edge              node {} (X6)
edge              node {} (X5)
(X3) edge              node {} (X5)
edge              node {} (X6)
(X4) edge              node {} (X1)
edge              node {} (Y2)
(X5) edge              node {} (X2)
(X6) edge              node {} (X3)
(W1) edge              node {} (X2)
(W2) edge              node {} (X5);
\end{tikzpicture}}
\caption{Graph Representation of the Open-Loop Six Nodes Network, Only One Disjoint Path}
\end{minipage}
\begin {minipage}{0.49\linewidth}
\scalebox{0.6}{\begin{tikzpicture}[->,>=stealth',shorten >=1pt,node distance=1.8cm, semithick]
\tikzstyle{every state}=[fill=none,draw=black,text=black,minimum size=1cm]

\node[state]         (X1)                    {$X_1$};
\node[state]         (X2) [right of=X1]      {$X_2$};
\node[state]         (X3) [right of=X2]      {$X_3$};
\node[state]         (X4) [below of=X1]       {$X_4$};
\node[state]         (X5) [below of=X2]       {$X_5$};
\node[state]         (X6) [below of=X3]       {$X_6$};
\node[state]         (Y1) [left of=X1]       {$Y_1$};
\node[state]         (Y2) [left of=X4]       {$Y_2$};
\node[state]         (W1) [above right of=X2] {$W_1$};
\node[state]         (W2) [below right of=X5] {$W_2$};
\path (X1) edge              node {} (X4)
edge              node {} (X5)
edge              node {} (Y1)
(X2) edge              node {} (X4)
edge              node {} (X6)
edge              node {} (X5)
edge              node {} (X1)
(X3) edge              node {} (X5)
edge              node {} (X6)
(X4) edge              node {} (X1)
edge              node {} (Y2)
(X5) edge              node {} (X2)
edge              node {} (X4)
(X6) edge              node {} (X3)
(W1) edge              node {} (X2)
(W2) edge              node {} (X5);
\end{tikzpicture}}
\caption{Graph Representation of the Closed-Loop Six Nodes Network, Two Disjoint Paths}
\end{minipage}
\end{figure}
\end{frame}



\begin{frame}{A Numerical Example}
\begin{itemize}
\item From the open-loop transfer matrix, we notice that if making the attack
$\begin{bmatrix}[c]
u_1\\
u_2
\end{bmatrix}=
\begin{bmatrix}[c]
-\frac{1}{s}\\
1
\end{bmatrix}
U(s)$, then the attack cannot be detected from $y$.
\item Specifically, I made $u_1(t)=-10-10cos(t)$, $u_2(t)=sin(t)$. Let's see the simulation results.

\end{itemize}
\end{frame}
\begin{frame}{A Numerical Example}
\includegraphics[width=4in, height=3in]{fault.eps}
\end{frame}


\section{Monitor Design}
\begin{frame}{Monitor Design}

If the following holds, 
\begin{itemize}
\item $w(0)=x(0)$... {\color{red} if not, asymptotically converging}
\item $(E, A_D+GC)$ is regular and Hurwitz {\color{blue} modified, $A_D=\text{blkdiag}(A_1, A_2, ...A_N)$}
\item Must know all regular control input ...{\color{red} assume zero in the paper}
\item $(E_i,A_i,C_i)$ must be observable...{\color{red}this is very hard to guarantee} {\color{blue} modified}
\item $(E_i,A_i)$ is regular.. {\color{blue} modified}
\item $|sE-A|$ does not vanish for all $s$
\item the initial condition x(0) is consistent ... {\color{red}satisfy the algebraic equation, $Ax(0)+Bu(0)\in Im(E)$}
\item the unknown attack $u_K(t)$ is sufficiently smooth ... {\color{red} in case of impulsive input}
\item $\rho((j\omega E-A_D-GC)^-1A_C)<1$ for all $\omega \in R$ ... {\color{blue} extra}
\item All attacks are detectable
\end{itemize}
Then $r(t) = 0$ if and only if $u(t) = 0 $.
\end{frame}

\begin{frame}{Monitor Design}

How to do?%%/footcite{dorfler2013continuous}
\begin{itemize}
\item collect samples of $y_i(t), \forall i$.
\item set an initial stage $w_i(t) \forall i$.
\item set $k:=k+1$, compute $w_i^{(k)}(t)$, $t\in[0,T]$, by integrating\\
$E_i \dot{w}_i^{(k)}(t) = (A_i+G_iC_i)w_i^{(k)}(t) + \sum_{j\neq i}A_{ij}w_j^{(k-1)}(t)- G_iy_i(t)$.
\item transmit $w_i^{(k)}(t)$ to control center $j$ if $A_{ij}\neq 0$.
\item update $w_j^{(k)}(t)$ with signal received from control center $j$.
\end{itemize}
For sufficient large $k$, all $r_i(t)$ will goes to zero if there is no attack.
This is just in horizon $[0,T]$, we can have more time horizon.$[T,2T],....$.
\end{frame}

\end{document}