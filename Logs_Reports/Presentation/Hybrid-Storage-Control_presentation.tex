\documentclass{beamer}
\makeatletter
\renewcommand*\env@matrix[1][*\c@MaxMatrixCols c]{%
	\hskip -\arraycolsep
	\let\@ifnextchar\new@ifnextchar
	\array{#1}}
\makeatother
\usepackage[style=verbose,backend=biber]{biblatex}
\addbibresource{ref.bib}
\setbeamerfont{footnote}{size=\tiny}
\beamertemplatenavigationsymbolsempty

\usetheme{Boadilla}

%PACKAGES
\usepackage{color}
\usepackage{caption}
\usepackage{graphicx}
\usepackage{amsmath}
%\usepackage{parskip}
\usepackage{amssymb}
%\usepackage{natbib}
\usepackage[utf8]{inputenc}
\usepackage{pgf}
\usepackage{tikz} %Use Tiks to plot graph
\usetikzlibrary{arrows,automata}
%figure package
\usepackage{float}


%START PRESENTATION DETAILS
\title{Control of Hybrid Energy Storage}

\DeclareMathOperator*{\argmin}{argmin}
\DeclareMathOperator{\E}{\mathbb{E}}
\newcommand{\norm}[1]{\left\lVert#1\right\rVert}
% correct bad hyphenation here
\hyphenation{op-tical net-works semi-conduc-tor}


\author{Alok Deshpande}


\date{February 7, 2018}

\AtBeginSection[]
{
	\begin{frame}<beamer>{Outline}
	\tableofcontents[currentsection,currentsubsection]
\end{frame}
}


%START PRESENTATION
\begin{document}

\begin{frame}
\titlepage
\end{frame}

%AUTOMATED outline
\begin{frame}{Outline}
\tableofcontents
% You might wish to add the option [pausesections]
\end{frame}
% Section and subsections will appear in the presentation overview
% and table of contents.

\section{Background}
\begin{frame}{Background}
\begin{itemize}
\item Battery life is critical to EV operation (e.g. driving time) 
\item Common design choice is hybrid power supply \footcite{thounthong2009energy}
\begin{itemize}
\item typically battery and supercapacitor
\item improves battery life
\end{itemize}
\item Reason: each is best suited for specific energy demand (load) profile \footcite{thounthong2009energy}
\begin{itemize}
\item Supercapacitor: high power output; limited energy capacity
\item Battery: opposite
\end{itemize}
\end{itemize}
\end{frame}
\begin{frame}{Problem}
\begin{itemize}
\item Goal: optimal sequence of energy discharges from each storage
\item Requirements:
\begin{itemize}
\item Net discharge = random demand
\item Minimize power transfer costs for each storage device
\item Minimize unnecessary energy transfers (losses)
\end{itemize}
\end{itemize}
\end{frame}

\section{Model}

\begin{frame}{Definitions}
Have separate charging and discharging (most general), and differing levels of leakage
\begin{itemize}
\item \textbf{Indices}\\
$t$: discrete time step index\\
$j$: storage device number (1: battery, 2: supercapacitor)\\
\item \textbf{Parameters}\\
$\alpha^{C}$: charging efficiency\\
$\alpha^{D}$: discharging efficiency\\
$\beta$: storage efficiency factor (constant)\\
$N$: number of steps (DP horizon)\\
$K$: cost weighting factor for rate (relative to cost of energy loss)\\
\item \textbf{Variables}\\
$L$: load energy demand (random)\\
$E$: energy state of storage device\\
$D$: energy released by discharging\\
$C$: energy consumed by charging\\
$J$: cost function

\end{itemize}
\end{frame}

\begin{frame}{Constraints}
\begin{itemize}
\item Supply-demand balance: 
\begin{equation} \label{eq:BalanceEqn}\left[D_{1}(t)\right] + \left[D_{2}(t) - C_{2}(t)\right] = L(t) \end{equation}

\item Bounds on stored energy: 
\begin{equation}E_{j}^{min}\leq E_{j}(t)\leq E_{j}^{max}\end{equation}
\item Bounds on charging:
\begin{equation}0\leq C_{j}(t)\leq C_{j}^{max}\end{equation}
\item Bounds on discharging:
\begin{equation}0\leq D_{j}(t)\leq D_{j}^{max}\end{equation}
\end{itemize}
\end{frame}

\begin{frame}{Recursive State Equations}
\begin{itemize}
\item Based on common formulation for a single storage \footcite{su2013modeling}
\item Battery can only supply load or supercapacitor (no charging):
\begin{equation}\label{eq:StateEq1}E_{1}(t+1)=\beta_{1}E_{1}(t)+\left[-\frac{1}{\alpha_{1}^{D}}D_{1}(t)\right] \end{equation}
\item Supercapacitor can only supply load, but can be charged by battery:
\begin{equation}\label{eq:StateEq2}E_{2}(t+1)=\beta_{2}E_{2}(t)+\left[\alpha_{2}^{C}[D_{1}(t)+D_{2}(t)-L(t)]-\frac{1}{\alpha_{2}^{D}}D_{2}(t)\right]\end{equation}
\begin{figure}
\includegraphics[width=2in, height=1in]{Energy_flows_diagram.PNG}
\end{figure}
\end{itemize}
\end{frame}

\begin{frame}{Cost Function}
\begin{itemize}
\item Minimize discharge rate
for the first storage device (battery):
\begin{equation}J_{rate}=min\left[\sum_{t=0}^{N-1}K\left[D_{1}(t)\right]^{2}\right]\end{equation}
\begin{itemize}
\item Quadratic cost in time penalizes more for high excursions
\item Common formulation \footcite{bambang2014energy}
\item Alternatives:
\begin{itemize}
\item Linear cost in time (very simplistic) \footcite{bordin2017linear} \footcite{ru2013storage}
\item Convex cost in number of cycles (too complex) \footcite{shi2017optimal}
\end{itemize}
\end{itemize}

\item Minimize energy loss due to transfers between storages

\noindent
\scalebox{0.7}{
\begin{equation}
J_{loss}=\mathop{\E}_{L(t)} \Biggl\{min\left[\sum_{t=0}^{N-1}
(1-\alpha_{1}^{D})D_{1}(t)+
(1-\alpha_{2}^{C})[D_{1}(t)+D_{2}(t)-L(t)]+
(1-\alpha_{2}^{D})D_{2}(t)
\right]\Biggr\}
\end{equation}
}
%\begin{itemize}
%	 \item Has "natural" loss weightings
%\end{itemize}

\end{itemize}
\end{frame}

\begin{frame}{Cost Function (cont.)}
\begin{itemize}
\item Combined cost function
\begin{multline}J=\mathop{\E}_{\substack{ L(t) \\ \forall t\in{0\dots N-1} }} \Biggl\{min\left[\sum_{t=0}^{N-1}K\left[D_{1}(t)\right]^{2} + (1-\alpha_{1}^{D})D_{1}(t)+\\
(1-\alpha_{2}^{C})[D_{1}(t)+D_{2}(t)-L(t)]+
(1-\alpha_{2}^{D})D_{2}(t)\right]\Biggr\}
\end{multline}

This is of the form:
\begin{equation}J=\mathop{\E}_{w(t)} \Biggl\{\min_{u}\left[\sum_{t=0}^{N-1}g(x(t),u(t),w(t))\right]\Biggr\}\end{equation}
for state $x(t)$, control $u(t)$ and random perturbation $w(t)$. Here, $g(\cdot)$ is the stage cost.
\end{itemize}
\end{frame}

\section{Finite Horizon DP}
\begin{frame}{Bellman Equation formulation}
\begin{itemize}
\item Can apply dynamic programming to find sequence of optimal controls $u^{*}(t)$
\item Reformulate as Bellman’s equation:
\begin{multline}
J_{t}[x(t),w(t)]=\min_{u} g(x(t),u(t),w(t)) + \\\mathop{\E}_{w(t+1)} \{J_{t+1}[f(x(t),u(t),w(t)),w(t+1)]\}
\end{multline}
so, can solve recursively for N iterations
\item In this case:
\begin{multline}
J_{t}[E_{1}(t),E_{2}(t),L(t)] = \min_{D_{1},D_{2}}
(1-\alpha_{1}^{D})D_{1}(t) 
+ K[D_{1}(t)]^{2}\\
+(1-\alpha_{2}^{D})[D_{2}(t)]	  +(1-\alpha_{2}^{C})[D_{1}(t)+D_{2}(t)-L(t)]\\
+\mathop{\E}_{L(t+1)}\{J_{t+1}[f_{1}(E_{1}(t),D_{1}(t)), f_{2}(E_{2}(t),D_{1}(t),D_{2}(t),L(t)), L(t+1)]\}
\end{multline}
\end{itemize}
\end{frame}

\begin{frame}{Solution}
\begin{itemize}
\item Choose terminal cost of 0
\begin{itemize}
\item assume vehicle reaches destination
\item hence, assume not fully discharged in $\textless$N stages
\end{itemize}
\item Then, costs are $\cdots$
\begin{itemize}
\item $N^{th}$ stage:
\begin{equation}
J_{N-1}[x(N-1),w(N-1)]=0
\end{equation}		
\item $(N-1)^{th}$ stage: \\
%\scalebox{0.8}{
\begin{multline}
J_{N-2}[x(N-2),w(N-2)]=\min_{u} g(x(N-2),u(N-2),w(N-2))+\\ \mathop{\E}_{w(N-1)}\{0\}
\end{multline}
%	}
%	\newline
\item Initial stage (where $f(\cdot)$ is the general recursive state equation): \\	
\begin{multline}
J=J_{0}[x(0),w(0)]=\min_{u} g(x(0),u(0),w(0))\\
+\left[\sum_{t=1}^{N-1}\mathop{\E}_{w(t)} \{
g( f(x(t-1),u(t-1),w(t-1)) ,u(t),w(t))
\}\right]
\end{multline}
\end{itemize}
\end{itemize}
\end{frame}

\section{FHDP Results}
\begin{frame}{Sample Numerical Solutions\footnote{E1: battery, E2: supercapacitor}}
After finding all control policies by iterating backwards ("offline"), ran with sample load sequences to find unique policies "online".\newline\newline

\begin{minipage}{0.5\textwidth}
\begin{figure}
\textbf{Constant demand}: battery supplies load
\includegraphics[width=2.5in,height=1.5in]{FHDP_Constant_Demand.jpg}
\end{figure}
\end{minipage}\begin{minipage}{0.5\textwidth}
\begin{figure}
\textbf{Ramp demand}: battery supplies, supercapacitor supplements
\includegraphics[width=2.5in, height=1.5in]{FHDP_Ramp_Demand.jpg}
\end{figure}
\end{minipage}

\end{frame}

\begin{frame}{Sample Numerical Solutions (cont.)}
\begin{figure}
\textbf{Fluctuating demand}: initially, \textit{still} optimal for battery to supply load
\includegraphics[width=2.5in, height=1.5in]{FHDP_Varying_NoEndDemand.jpg}
\includegraphics[width=2.5in, height=1.5in]{FHDP_Varying_wEndDemand.jpg}
\end{figure}
\begin{itemize}
\item Lowest cost for supplying using supercapacitor at \textit{end} of sequence
\begin{itemize}
\item \textbf{Reason}: when supercapacitor is \textit{symmetrically} efficient, negligible effect of $D_{2}$!
\end{itemize}
\item Battery is seen to charge supercapacitor
\end{itemize}
\end{frame}

\begin{frame}{Sample Numerical Solutions (cont.)}
\textbf{Demand, with decreased battery efficiency}: for lower discharging efficiency (50\%), use supercapacitor more
\begin{figure}
\includegraphics[width=3in, height=1.8in]{FHDP_Varying_LowBattDisEff.jpg}
\end{figure}
\begin{itemize}
\item \textbf{Other load distributions}: tried normal distribution of demand, instead of uniform (above). Same conclusions.
\end{itemize}
\end{frame}



\section{Infinite Horizon DP}
\begin{frame}{Discounted Cost Problem formulation}
\begin{itemize}
\item More appropriate for long duration: Infinite Horizon
\item When cost becomes constant over time, problem is complete (constant policy)
\item For N$\to\infty$, this will happen for a “discounted” cost-to-go:
\begin{multline}
J_{0}[x(0),w(0)]=\min_{u} g(x(0),u(0),w(0))\\
+\left[\sum_{t=1}^{N-1}\mathop{\E}_{w(t)} \{
\alpha^{t}*g( f(x(t-1),u(t-1),w(t-1)) ,u(t),w(t))
\}\right]
\end{multline}
where $0<\alpha<1$ is the discount factor
\item In the limit, terms with coefficients $\alpha^{N},N\to\infty$ become negligible, so:
\begin{multline}
J^{*}[x(N-2),w(N-2)]=\min_{u} g(x(N-2),u(N-2),w(N-2))\\
+\mathop{\E}_{w(N-1)} \{\alpha J^{*}[f(x(N-2),u(N-2),w(N-2)),w(N-1)]
\}
\end{multline}
where $J^{*}$ is the converged cost function
\end{itemize}
\end{frame}

\begin{frame}{Cost convergence}
\begin{itemize}
\item Even after a limited number of iterations, the cost starts to converge
\item The cost is said to have converged after N iterations when:
\begin{equation}
\norm{J[x(N),w(N)]-J[x(N+1),w(N+1)]} < \epsilon
\end{equation}
\item Example, based on numerical solution:
\end{itemize}
\begin{figure}
\includegraphics[width=3in, height=1.8in]{Convergence_of_cost.jpg}
\caption{Convergence, for $\epsilon=10$}
\end{figure}
\end{frame}

\begin{frame}{Value Iteration solution}
\begin{itemize}
\item The infinite-horizon policy  ($\pi^{*}=\{u^{*}\}$) can be approximated in N iterations
\begin{itemize}
\item $\pi^{*}$ is set of optimal controls after N iterations
\end{itemize}
\item Follow two-step process to determine control sequence:
\begin{enumerate}
\item Find $\pi^{*}$ at time N (running "offline")
\item Apply the policy for T additional iterations to apply the specific control sequence from $\pi^{*}$ (in response to a real-time load sequence)
\end{enumerate}
\item This is "Value Iteration"
\end{itemize}
\end{frame}

\section{IHDP Results}
\begin{frame}{Sample Numerical Solutions\footnote{E1: battery, E2: supercapacitor}}
\begin{minipage}{0.5\textwidth}
\begin{figure}
\textbf{Constant demand}: battery supplies load
\includegraphics[width=2.5in,height=1.5in]{IHDP_Constant_Demand.jpg}
\end{figure}
\end{minipage}\begin{minipage}{0.5\textwidth}
\begin{figure}
\textbf{Ramp demand}: battery supplies, supercapacitor supplements
\includegraphics[width=2.5in, height=1.5in]{IHDP_Ramp_Demand.JPG}
\end{figure}
\end{minipage}
\end{frame}

\begin{frame}{Sample Numerical Solutions (cont.)}
\textbf{Fluctuating demand}: supercapacitor is used after initially using the battery
\begin{figure}
\includegraphics[width=3in, height=1.8in]{IHDP_Varying_Demand1.jpg}
%\includegraphics[width=2.5in, height=1.5in]{IHDP_Varying_Demand2.jpg}
\end{figure}

\begin{itemize}
\item \textbf{Overall}: same conclusions as FHDP
\end{itemize}
%\item In some cases, supercapacitor is never used
\end{frame}

\section{Issue with Exact DP}
\begin{frame}{Dimensionality Issue}
\begin{itemize}
\item When finding policy, need to calculate cost for \textit{every} combination of discrete state components
\begin{itemize}
\item For $M$ states of $E_{1}$, $N$ states of $E_{2}$, $P$ states of $L$, have $M \times N \times P$ combinations
\item This can be very large!
\end{itemize}
\item Testing program run time (with P=M+N):

\begin{minipage}{0.3\textwidth}
\begin{figure}
\includegraphics[width=1.3in, height=0.5in]{Convergence_Time_Test1.JPG}
\caption{M=10,\\N=5}
\end{figure}
\end{minipage}
\begin{minipage}{0.3\textwidth}
\begin{figure}
\includegraphics[width=1.3in, height=0.5in]{Convergence_Time_Test2.JPG}
\caption{M=20,\\N=10}
\end{figure}
\end{minipage}
\begin{minipage}{0.3\textwidth}
\begin{figure}
\includegraphics[width=1.3in, height=0.5in]{Convergence_Time_Test3.JPG}
\caption{M=30,\\N=15}
\end{figure}
\end{minipage}

\item Do not use exact DP!
\end{itemize}
\end{frame}

\section{Alternatives and Future Work}
\begin{frame}{Alternative: Approximative DP}
\begin{itemize}
\item Possible cost at each state, $J(\textbf{x})$ forms a 3D matrix
\item Goal: avoid having to recompute cost each time
\item Approximate the cost structure\footcite{bertsekas1995dynamic} by “fitting” $J(\textbf{x})$ to basis functions in matrix $\Phi$,  with linear weightings in vector $\textbf{r}$:
\begin{equation}
J(\textbf{x})\approx\Phi\textbf{r}
\end{equation}
Here $\textbf{x}=(E_{1},E_{2},L)$ and
\begin{math}
\Phi_{i}=[f(E_{1}),g(E_{2}),h(L),\cdots]
\end{math}
in the $i^{th}$ row of $\Phi$
\item The functions $f(\cdot), g(\cdot), h(\cdot), \cdots$ are basis functions to be determined
\item Extrapolate to find costs in other states

\end{itemize}
\end{frame}

\begin{frame}{Approximative DP (cont.)}
\begin{figure}
\includegraphics[width=4.5in, height=2.7in]{ValueFunc_Surfaces_EX.jpg}
\caption{Illustration of Cost Matrices (3 iterations)}
\end{figure}
\end{frame}


\begin{frame}{Conclusion and Future Work}
\begin{itemize}
\item DP can be applied to determine an optimal control policy for two complementary storage devices
\item Model for single-storage case extended to two storages
\item If probability distribution for load and starting state are known, can determine optimal policy in real-time
\item To simulate a real driving duration, solve the infinite horizon problem
\item Future work:
\begin{itemize}
\item Find basis functions for ADP
\item “Round” states that fall outside discretization
\item Regenerative braking
\end{itemize}

\end{itemize}
\end{frame}
\end{document}