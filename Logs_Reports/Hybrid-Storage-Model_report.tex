\documentclass{article}
\usepackage[utf8]{inputenc}

\title{Hybrid Storage Model}
\author{Alok Deshpande}
\date{October 30, 2017}

\usepackage{natbib}
\usepackage{graphicx}
\usepackage{amsmath}
\usepackage{amssymb}

\DeclareMathOperator{\E}{\mathbb{E}}

\begin{document}
	
	\maketitle
	
	\section{Introduction}
	This model describes the energy flows in a two- storage system. It will be formulated as a dynamic programming problem.
	
	\section{Model}
	
	\subsection{Definitions}
	\begin{itemize}
		\item Indices\\
		$i$: discrete time step index\\
		$j$: storage device number (1: battery, 2: supercapacitor)\\
		\item Parameters\\
		$\alpha^{C}$: charging efficiency\\
		$\alpha^{D}$: discharging efficiency\\
		$\beta$: storage efficiency factor (constant)\\
		$N$: number of steps (DP horizon)\\
		$K$: cost weighting factor for rate (relative to cost of energy loss)\\
		\item Variables\\
		$L$: load energy demand (Random Variable!!)\\
		$E$: energy state of storage device\\
		$D$: energy released by discharging (AFTER loss)\\
		$C$: energy consumed by charging (BEFORE loss)\\
		$J$: value function
	\end{itemize}
	
	NOTE: $C_{1}$ does not exist because not possible to charge the battery while driving. (Assuming no regenerative braking at the moment.)
	
	\subsection{Constraints}
	\begin{itemize}
		\item Supply-demand balance: 
		\begin{equation} \label{eq:BalanceEqn}\left[D_{1}(i)\right] + \left[D_{2}(i) - C_{2}(i)\right] = L(i) \end{equation}
						
		\item Bounds on stored energy: 
		\begin{equation}E_{j}^{min}\leq E_{j}(i)\leq E_{j}^{max}\end{equation}
		\item Bounds on charging:
		\begin{equation}0\leq C_{2}(i)\leq C_{2}^{max}\end{equation}
		\item Bounds on discharging:
		\begin{equation}0\leq D_{j}(i)\leq D_{j}^{max}\end{equation}
	\end{itemize}

	\subsection{Recursive State Equations}
	 The state of the system is the energy in a storage device ($E_{j}(i)$). This evolves according to the charging and discharging of the storage device, which is the control.
	 
	 The following recursive equations describe the changes in the state, including due to constant leakage loss:
	 
	 \begin{equation}\label{eq:StateEq1}E_{1}(i+1)=\beta_{1}E_{1}(i)+\left[-\frac{1}{\alpha_{1}^{D}}D_{1}(i)\right] \end{equation}
	 
	 \begin{equation}\label{eq:StateEq2}E_{2}(i+1)=\beta_{2}E_{2}(i)+\left[\alpha_{2}^{C}C_{2}(i)-\frac{1}{\alpha_{2}^{D}}D_{2}(i)\right]\end{equation}
	 
	 Substituting \eqref{eq:BalanceEqn} into \eqref{eq:StateEq2}, one obtains:
	 
	 \begin{equation}\label{eq:DPStateEq2}E_{2}(i+1)=\beta_{2}E_{2}(i)+\left[\alpha_{2}^{C}[D_{1}(i)+D_{2}(i)-L(i)]-\frac{1}{\alpha_{2}^{D}}D_{2}(i)\right]\end{equation}
	 

	\subsection{Objective Function}
	\begin{itemize}
		\item Minimize charge and discharge rates
		 for the first storage device (battery):
		 \begin{equation}J_{rate}=min\left[\sum_{i=0}^{N}K\left[D_{1}(i)\right]^{2}\right]\end{equation}
 		\item Minimize power loss due to energy transfers
		 \begin{equation}J_{loss}=min\left[\sum_{i=0}^{N}
		 (1-\alpha_{1}^{D})D_{1}(i)+
		 (1-\alpha_{2}^{C})C_{2}(i)+
 		 (1-\alpha_{2}^{D})D_{2}(i)
		  \right]\end{equation}
	\end{itemize}
	
	Hence, the cost functions are as follow:
	  
	\begin{itemize}
		  \item Battery storage:\\
		  \begin{multline}
		  J_{i}[E_{1}(i)] = \mathop{\mathbb{E}}_{L(i)}
		  \{\min_{D_{1},D_{2}}
		  (1-\alpha_{1}^{D})D_{1}(i) 
		  	+ K[D_{1}(i)]^{2}
		  	+(1-\alpha_{2}^{D})[D_{2}(i)]\\	  +(1-\alpha_{2}^{C})[D_{1}(i)+D_{2}(i)-E\{L(i)\}]
		  	+\mathop{\mathbb{E}}_{L(i+1)}\{J_{i+1}[f_{1}(E_{1}(i),D_{1}(i)),L(i+1)]\}\}
		  \end{multline}
		  
		  where $f_{1}(\cdot)$ is \eqref{eq:StateEq1}, the state equation for the battery.
		  
		  \item Supercapacitor storage:\\
		  \begin{multline}
		  J_{i}[E_{2}(i)] = \mathop{\mathbb{E}}_{L(i)} \{\min_{D_{1},D_{2}}
		  (1-\alpha_{1}^{D})D_{1}(i) 
		  	+ K[D_{1}(i)]^{2}
		  	+(1-\alpha_{2}^{D})[D_{2}(i)]\\	  +(1-\alpha_{2}^{C})[D_{1}(i)+D_{2}(i)-L(i)]
		  	+\mathop{\mathbb{E}}_{L(i+1)} \{J_{i+1}[f_{2}(E_{2}(i),D_{2}(i)),L(i+1)]\}\}
		  \end{multline}
		  
		  where $f_{2}(\cdot)$ is \eqref{eq:DPStateEq2}, the state equation for the supercapacitor.
	\end{itemize}
	
	\bibliographystyle{plain}
	\bibliography{references}
\end{document}
